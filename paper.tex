\documentclass[a4paper]{article}                                     
\usepackage{url,xspace,epsfig,etoolbox}
\usepackage{natbib}
\usepackage{fullpage}
\usepackage[font=bf]{caption}
\usepackage{mdwlist}

% No page numbers
\pagestyle{empty}

% Use 'Arial' font
\usepackage{helvet}
\renewcommand{\familydefault}{\sfdefault}

% Allow point-size specification of font
\usepackage{fix-cm}

% Allow setting margins on the abstract
\usepackage{changepage}

\newcommand{\projecttitle}{Definition and directions of narrative-rich games}

% Prevent indentation of new paragraphs
\setlength{\parindent}{0pt}

% Put space after each paragraph. Note that this matches section titles too,
% and the workaround is below.
\setlength{\parskip}{\baselineskip}

% Allow easy overriding of section and subsection commands
\usepackage{titlesec}
\titleformat{\section}{\bfseries\fontsize{12}{14.4}\selectfont}{}{0em}{}
\titleformat{\subsection}{\bfseries\fontsize{11}{13.2}\selectfont}{}{0em}{}

% No blank line after section titles.
\titlespacing{\section}{0pt}{0pt}{-\baselineskip}
\titlespacing{\subsection}{0pt}{0pt}{-\baselineskip}

% No blank spaces between references
\setlength{\bibsep}{0pt}

% Hanging indent of 0.5 inches in references
\setlength{\bibhang}{0.5in}

% Keep all the screenshot figures the same width
\newcommand{\screenshotwidth}{2.5in}

% Is this blind or not?
\newtoggle{blind}
\toggletrue{blind}

\begin{document}

% Title is bold, centered, 14pt.
% Standard baseline skip is 1.2*fontsize, or 16.8 here.
\begin{center}
\fontsize{14}{16.8}\selectfont
\bf \projecttitle
\end{center}

% Gleaning author placement information from the existing proceedings.
\vspace{-0.25in}
\begin{center}
\iftoggle{blind}{
 % Nothing here: skip authors for blind review
}{
Paul Gestwicki, others, Ball State University\\
pvgestwicki@bsu.edu\\
}
\end{center}


% Abstract: 150 words or less, extra 0.5in left and right margin.
% Note that all text is justified.
%
\begin{adjustwidth}{0.5in}{0.5in}
  \textbf{Abstract:} \textit{Player-narration games} are a promising
  genre for transformative games. This genre can be defined formal and
  dramatic characteristics, and we identify tabletop games that belong
  to the genre.  Adding technological components to the game affords
  new interaction modes as well as convenient digital distribution. We
  describe our team's pilot project at creating a technology-enhanced
  player-narration game to improve cultural empathy among players aged
  10--14.  This pilot project highlights several complications with
  this genre, including thematic, technical, and production
  challenges.  However, the pilot project justifies continued
  experimentation, development, and formal assessment within the
  genre.
\end{adjustwidth}

\section{Introduction}

Content goes here.
We might cite \citet{Koster2013}, but this example is really just a placeholder
so the bibtex portion of the build script succeeds.


\section{Acknowledgments}
\textit{The team, organizational, and community partner
  acknowledgments are hidden for blind review.}

\bibliographystyle{apalike}
\bibliography{references}

\end{document} 